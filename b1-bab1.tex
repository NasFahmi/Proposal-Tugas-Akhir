%==================================================================
% Ini adalah bab 1
% Silahkan edit sesuai kebutuhan, baik menambah atau mengurangi \section, \subsection
%==================================================================

\chapter[PENDAHULUAN]{\\ PENDAHULUAN}

\section{Latar Belakang}
Usaha Mikro, Kecil, dan Menengah (UMKM) merupakan salah satu pilar utama dalam perekonomian Indonesia. Berdasarkan beberapa studi nasional, UMKM berkontribusi besar terhadap Produk Domestik Bruto (PDB) serta memiliki peran penting dalam penyerapan tenaga kerja. Namun, meskipun kontribusinya besar, tingkat literasi digital UMKM di Indonesia masih tergolong rendah, terutama dalam pemanfaatan teknologi informasi untuk analisis pasar dan pengambilan keputusan berbasis data. Penelitian oleh \citep{Irianto2022} menyebutkan bahwa banyak UMKM belum mampu memanfaatkan teknologi digital secara optimal karena keterbatasan pengetahuan, akses, dan kemampuan mengelola data.

Di era digital saat ini, perilaku konsumen telah berubah secara signifikan. Media sosial seperti Instagram, TikTok, dan Facebook tidak hanya menjadi saluran pemasaran, tetapi juga menjadi ruang interaksi antara konsumen dan pelaku usaha. Studi oleh \citep{Trulline2021} menunjukkan bahwa UMKM semakin bergantung pada media sosial untuk mempromosikan produk, membangun reputasi, dan memahami minat pasar. Komentar konsumen pada postingan media sosial mencerminkan persepsi publik terhadap produk atau layanan UMKM, sehingga dapat dimanfaatkan sebagai indikator kepuasan dan reputasi merek.

Namun, data komentar media sosial bersifat tidak terstruktur, jumlahnya besar, dan berubah secara dinamis, sehingga sulit dianalisis tanpa bantuan teknologi. Sejalan dengan \citep{Joseph2024}, Sentiment Analysis pada media sosial merupakan bidang penelitian yang berkembang pesat dan melibatkan penggunaan Natural Language Processing (NLP), text analysis, dan computational linguistics untuk mengidentifikasi serta mengekstraksi informasi subjektif dari data teks. Pendekatan NLP tersebut memungkinkan komentar tidak terstruktur diproses menjadi kategori sentimen yang dapat diinterpretasikan, baik positif, negatif, maupun netral. Dengan demikian, hasil analisis sentimen dari proses NLP dapat disajikan dalam bentuk dashboard interaktif agar UMKM dapat memahami tren sentimen, isu yang sering muncul, serta persepsi pelanggan secara lebih cepat dan intuitif.

Dalam pengembangan dashboard analitik berbasis web, tantangan tidak hanya terletak pada penyajian visualisasi data, tetapi juga pada pengelolaan alur data di sisi frontend. Dashboard analitik merupakan aplikasi yang bersifat data-driven, di mana berbagai komponen antarmuka seperti grafik, tabel, dan indikator bergantung pada data yang sama dan diperbarui secara berkala melalui API. Jika pengambilan data dilakukan secara langsung pada setiap komponen tanpa arsitektur pengelolaan data yang terstruktur, maka dapat muncul berbagai permasalahan, seperti permintaan API berulang, inkonsistensi data antar-komponen, serta kesulitan dalam pemeliharaan kode aplikasi.

Pendekatan arsitektur Client Data Layer hadir sebagai solusi untuk mengelola data yang bersumber dari server secara terpusat di sisi klien. Dengan adanya Client Data Layer, proses pengambilan, penyimpanan sementara, dan sinkronisasi data dapat dilakukan secara terstruktur, sehingga komponen antarmuka tidak perlu berinteraksi langsung dengan API. Salah satu pustaka yang mendukung pendekatan ini adalah TanStack Query, yang dirancang untuk mengelola server state secara efisien melalui mekanisme caching, deduplikasi permintaan, serta sinkronisasi data antar-komponen.

Sejumlah penelitian terdahulu menunjukkan bahwa TanStack Query memiliki keunggulan dalam pengelolaan data asinkron pada aplikasi frontend dibandingkan pendekatan pengelolaan data konvensional, khususnya pada aplikasi yang bersifat data-driven dan menampilkan data dalam jumlah besar \citep{Luz2025}. Penelitian yang membandingkan TanStack Query dengan pustaka state management lain juga melaporkan adanya peningkatan efisiensi pengelolaan server state dan pengurangan permintaan data yang tidak diperlukan \citep{Micheal2025}. Namun demikian, sebagian besar penelitian tersebut masih berfokus pada aspek teknis pustaka atau perbandingan antar-library, dan belum secara spesifik mengkaji penerapan TanStack Query sebagai Client Data Layer pada studi kasus dashboard analitik.

Berdasarkan permasalahan tersebut, penelitian ini muncul dari hipotesis bahwa penerapan arsitektur Client Data Layer menggunakan TanStack Query pada aplikasi dashboard analitik mampu menghasilkan pengelolaan data frontend yang lebih terstruktur dan konsisten dibandingkan pendekatan pengelolaan data konvensional, yaitu pengambilan data API secara langsung pada komponen antarmuka tanpa mekanisme pengelolaan data terpusat. Perilaku tersebut diasumsikan tercermin melalui pemanfaatan mekanisme caching, pengendalian permintaan data dari API, serta konsistensi data yang ditampilkan pada berbagai komponen dashboard. Untuk menguji hipotesis tersebut, TanStack Query diterapkan pada studi kasus pengembangan Dashboard Analisis Sentimen UMKM dengan metode Fountain, dan hasilnya dievaluasi melalui observasi perilaku sistem menggunakan pendekatan blackbox testing berbasis log.

\section{Rumusan Masalah}
\begin{packed_enum}
  \item Bagaimana menerapkan arsitektur Client Data Layer menggunakan TanStack Query pada pengembangan Dashboard Analisis Sentimen UMKM?
  \item Bagaimana perilaku pengelolaan data frontend yang dihasilkan setelah penerapan TanStack Query ditinjau dari log permintaan data, mekanisme caching, dan konsistensi data antar-komponen dashboard?
\end{packed_enum}


\section{Tujuan}

\indent Penelitian ini bertujuan untuk menerapkan arsitektur Client Data Layer menggunakan TanStack Query pada studi kasus pengembangan Dashboard Analisis Sentimen UMKM. Penerapan arsitektur ini difokuskan pada pengelolaan data API di sisi frontend agar proses pengambilan dan penyajian data dapat dilakukan secara terstruktur.

Selain itu, penelitian ini bertujuan untuk mengevaluasi perilaku pengelolaan data frontend setelah penerapan TanStack Query melalui pendekatan blackbox testing. Evaluasi dilakukan dengan mengamati log sistem yang berkaitan dengan permintaan API, pemanfaatan mekanisme caching, serta sinkronisasi data antar-komponen dashboard. Melalui penerapan dan evaluasi tersebut, penelitian ini diharapkan dapat memberikan gambaran praktis mengenai implementasi arsitektur Client Data Layer pada aplikasi frontend berbasis React.

\section{Manfaat}

Manfaat Teoritis

\begin{packed_enum}
  \item Penelitian ini diharapkan dapat menambah literatur mengenai
  penerapan TanStack Query dalam arsitektur data layer pada studi kasus
  aplikasi dashboard.
\end{packed_enum}

Manfaat Praktis

\begin{packed_enum}
  \item \textbf{bagi UMKM}

  Penelitian ini menghasilkan sebuah dashboard analisis sentimen yang dapat membantu UMKM memahami persepsi konsumen berdasarkan data media sosial secara lebih terstruktur dan mudah dipahami.

  \item \textbf{Manfaat bagi Pengembang}

  Penelitian ini memberikan studi kasus penerapan arsitektur Client Data Layer menggunakan TanStack Query yang dapat dijadikan acuan dalam merancang pengelolaan data frontend pada aplikasi data-driven.

  \item \textbf{Manfaat bagi Akademisi}

  Penelitian ini dapat dijadikan referensi bagi penelitian sejenis yang membahas arsitektur frontend, pengelolaan server state, dan pengembangan dashboard analitik.
\end{packed_enum}


\section{Batasan}
Batasan proyek ditetapkan agar penelitian tetap terfokus pada tujuan yang telah dirumuskan. Adapun batasan proyek dalam Tugas Akhir ini adalah sebagai berikut:


\begin{packed_enum}
  \item Studi kasus penelitian difokuskan pada Dashboard Analisis Sentimen UMKM dengan ruang lingkup penelitian terbatas pada sisi frontend aplikasi.
  \item Penelitian hanya membahas penerapan arsitektur Client Data Layer menggunakan TanStack Query pada aplikasi frontend berbasis React.
  \item Proses analisis sentimen, termasuk pengumpulan data media sosial dan pemrosesan teks, tidak dibahas dalam penelitian ini dan sepenuhnya dilakukan pada sisi backend.
  \item Penelitian ini tidak melakukan perbandingan implementasi TanStack Query dengan pustaka atau framework state management lain
  \item Evaluasi sistem dilakukan menggunakan pendekatan blackbox testing dengan mengamati log sistem, tanpa melakukan pengujian performa numerik atau benchmarking mendalam.
  \item Hasil evaluasi disajikan dalam bentuk ringkasan log, tabel, dan visualisasi sederhana untuk menggambarkan perilaku sistem setelah penerapan TanStack Query.
  \item Penelitian ini tidak membahas aspek optimasi backend, keamanan aplikasi, pengujian beban, maupun pengujian kegunaan secara mendalam.
\end{packed_enum}
% \section{Keaslian Gagasan}
% Keaslian gagasan bertujuan untuk menekankan inovasi atau kontribusi unik yang ditawarkan oleh proyek ini. Bagian ini menjelaskan bagaimana proyek ini menawarkan pendekatan yang berbeda atau peningkatan dibandingkan dengan metode atau perangkat yang sudah ada. Keaslian gagasan dapat diperlihatkan melalui perbandingan dengan proyek atau produk serupa, menunjukkan perbedaan signifikan atau keunggulan yang dihadirkan oleh solusi yang diusulkan. Misalnya, peningkatan kinerja, efisiensi, atau kemudahan penggunaan yang dihasilkan dari metode atau pendekatan baru. Selain itu, bagian ini juga bisa mencakup penggunaan teknologi atau desain yang belum banyak diterapkan dalam konteks yang sama. Penekanan pada keaslian gagasan membantu menunjukkan bahwa proyek ini tidak hanya mengikuti pola yang sudah ada, tetapi juga menghadirkan sesuatu yang baru dan relevan.

% \section{Sistematika Penulisan}
% Sistematika penulisan memberikan panduan mengenai struktur dari keseluruhan laporan proyek ini, sehingga memudahkan pembaca dalam memahami alur isi laporan dari setiap bab. Bagian ini menjelaskan isi dari setiap bab secara singkat, mulai dari latar belakang hingga kesimpulan dan rekomendasi. Misalnya, BAB I membahas pendahuluan dan dasar pengembangan proyek, BAB II menguraikan tinjauan pustaka dan landasan teori, dan seterusnya. Dengan memberikan sistematika penulisan, pembaca dapat memahami bagaimana laporan ini disusun secara keseluruhan dan bagaimana setiap bab saling berkaitan dalam mencapai tujuan akhir proyek.