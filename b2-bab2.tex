%==================================================================
% Ini adalah bab 2
% Silahkan edit sesuai kebutuhan, baik menambah atau mengurangi \section, \subsection
%==================================================================

\chapter[TINJAUAN PUSTAKA]{\\ TINJAUAN PUSTAKA}

\section{Landasan Teori}

\subsection{UMKM dan Digitalisasi}

Usaha Mikro, Kecil, dan Menengah (UMKM) merupakan sektor usaha produktif yang memiliki peran penting dalam perekonomian, khususnya dalam menciptakan lapangan kerja dan mendorong pertumbuhan ekonomi lokal. UMKM umumnya memiliki karakteristik berupa skala usaha yang relatif kecil, keterbatasan modal, serta pengelolaan usaha yang masih sederhana. Dalam beberapa tahun terakhir, perkembangan teknologi digital telah mendorong UMKM untuk beradaptasi dengan perubahan lingkungan bisnis agar tetap mampu bersaing di tengah dinamika pasar yang semakin cepat.

Meskipun digitalisasi menawarkan berbagai peluang, UMKM masih menghadapi sejumlah tantangan dalam proses adopsinya. Tantangan tersebut meliputi rendahnya literasi digital, keterbatasan pemahaman dalam pemanfaatan teknologi informasi, serta kurangnya kemampuan dalam mengelola dan menganalisis data bisnis. Selain itu, salah satu kendala utama yang dihadapi UMKM adalah keterbatasan akses terhadap sumber daya yang dibutuhkan untuk mengembangkan usaha, seperti modal, informasi, dan teknologi \citep{Alviani2025}.

Permasalahan digitalisasi UMKM juga diperkuat oleh faktor demografis dan geografis. Menurut \citep{Sofyan2025}, rendahnya literasi digital, khususnya pada pelaku UMKM usia lanjut dan yang berada di wilayah terpencil, menjadi hambatan dalam proses transformasi digital. Selain itu, tidak semua UMKM memiliki perangkat pendukung dan akses jaringan internet yang memadai, sehingga proses digitalisasi belum dapat diterapkan secara merata.

Dalam konteks pemasaran, pemanfaatan teknologi digital, khususnya media sosial, telah menjadi salah satu sarana utama bagi UMKM untuk mempromosikan produk dan menjangkau konsumen secara lebih luas. Aktivitas pemasaran digital tersebut menghasilkan data interaksi konsumen dalam jumlah besar yang berpotensi memberikan informasi berharga mengenai persepsi dan preferensi pasar. Oleh karena itu, UMKM membutuhkan sistem informasi yang mampu mengolah dan menyajikan data tersebut secara terstruktur dan informatif. Keberadaan sistem informasi berbasis dashboard analitik menjadi penting untuk mendukung pengambilan keputusan berbasis data serta meningkatkan efektivitas strategi pemasaran UMKM di era digital \citep{Trulline2021}.


\subsection{Media Sosial sebagai Sumber Data}

Media sosial telah berkembang tidak hanya sebagai sarana komunikasi dan pemasaran, tetapi juga sebagai sumber data yang mencerminkan opini, persepsi, dan perilaku konsumen. Melalui media sosial, konsumen dapat mengekspresikan pengalaman serta keterlibatan mereka terhadap suatu merek melalui berbagai bentuk interaksi, seperti komentar, unggahan, tanda suka, dan aktivitas berbagi konten. Interaksi tersebut mencerminkan keterlibatan konsumen dan menghasilkan data yang bernilai untuk dianalisis lebih lanjut \citep{Mardhatilah2024}. Data yang dihasilkan dari media sosial umumnya bersifat tidak terstruktur dan terus bertambah secara dinamis, sehingga memerlukan sistem yang mampu mengelola dan menyajikan informasi tersebut secara terstruktur agar dapat dimanfaatkan secara optimal.

Data yang dihasilkan dari media sosial memiliki karakteristik bersifat tidak terstruktur, berjumlah besar, dan terus bertambah secara dinamis. Komentar dan ulasan konsumen umumnya berbentuk teks bebas yang mengandung opini subjektif, emosi, serta penilaian terhadap suatu produk atau layanan. Kondisi ini menyebabkan data media sosial sulit dianalisis secara manual. Oleh karena itu, diperlukan pendekatan sistematis untuk mengolah dan mengekstraksi informasi penting dari data tersebut agar dapat dimanfaatkan secara optimal.

Dalam konteks UMKM, data media sosial berpotensi memberikan wawasan penting terkait preferensi konsumen, tingkat kepuasan pelanggan, serta isu-isu yang sering muncul dalam interaksi publik. Informasi ini dapat dimanfaatkan sebagai dasar evaluasi strategi pemasaran dan pengambilan keputusan bisnis. Namun, tanpa dukungan sistem yang mampu mengelola dan menyajikan data secara terstruktur, potensi data media sosial tersebut sulit dimanfaatkan secara efektif.

Oleh karena itu, media sosial diposisikan sebagai salah satu sumber data utama dalam pengembangan sistem analitik bagi UMKM. Data yang diperoleh dari media sosial selanjutnya dapat diolah dan disajikan dalam bentuk informasi yang lebih ringkas dan mudah dipahami melalui dashboard analitik. Pendekatan ini memungkinkan UMKM untuk memantau persepsi konsumen dan dapat digunakan sebagai alat pengambilan keputusan.

\subsection{Analisis Sentimen pada Media Sosial}

Analisis sentimen merupakan pendekatan analitik yang digunakan untuk mengidentifikasi dan mengklasifikasikan opini atau sikap pengguna terhadap suatu objek, seperti produk, layanan, atau merek, berdasarkan data teks. Dalam konteks media sosial, analisis sentimen memanfaatkan komentar, ulasan, dan berbagai bentuk interaksi pengguna untuk menentukan kecenderungan sentimen yang umumnya dikategorikan ke dalam sentimen positif, negatif, atau netral. Pendekatan ini relevan karena media sosial menyediakan data opini konsumen yang bersifat spontan, terbuka, dan dihasilkan secara terus-menerus melalui interaksi pengguna.

Seiring dengan meningkatnya aktivitas pengguna di media sosial, volume data opini yang dihasilkan juga semakin besar dan bersifat dinamis. Data tersebut umumnya tidak terstruktur dan mengandung unsur subjektivitas, emosi, serta bahasa informal, sehingga sulit dianalisis secara manual. Oleh karena itu, analisis sentimen digunakan untuk menyederhanakan data teks yang kompleks menjadi informasi yang lebih ringkas dan terstruktur, sehingga dapat digunakan untuk memahami kecenderungan opini publik dan perilaku konsumen secara lebih sistematis.

Penelitian yang dilakukan oleh \citep{Fajarini2025} menunjukkan bahwa analisis sentimen berbasis data media sosial merupakan metode yang inovatif dan efektif dalam memprediksi tren pasar serta memahami perilaku konsumen. Dengan memanfaatkan data berskala besar yang dihasilkan secara real time oleh pengguna media sosial, analisis sentimen mampu mengenali pola opini publik, preferensi konsumen, serta perubahan tren yang relevan di berbagai sektor industri. Hasil penelitian tersebut juga menegaskan bahwa informasi yang dihasilkan dari analisis sentimen dapat memberikan wawasan yang bernilai untuk mendukung pengambilan keputusan bisnis berbasis data.

Lebih lanjut, \citep{Fajarini2025} menyatakan bahwa analisis sentimen berbasis media sosial menawarkan pendekatan yang lebih cepat dan efisien dibandingkan metode konvensional, seperti survei manual atau riset pasar tradisional, karena memungkinkan pemantauan opini publik secara berkelanjutan. Dengan demikian, analisis sentimen dipandang sebagai alat yang penting dalam mendukung strategi bisnis modern yang berbasis data dan responsif terhadap dinamika pasar.

Dalam penelitian ini, analisis sentimen diposisikan sebagai proses pengolahan data yang dilakukan pada sisi backend. Fokus penelitian tidak terletak pada metode atau algoritma analisis sentimen yang digunakan, melainkan pada pemanfaatan hasil analisis sentimen sebagai sumber data yang dikelola dan disajikan melalui dashboard analitik di sisi frontend. Hasil analisis sentimen tersebut digunakan sebagai input utama untuk mendukung penyajian informasi yang informatif, terstruktur, dan mudah dipahami oleh pengguna.


\subsection{Dashboard Analitik dan Visualisasi Data}

Dashboard analitik merupakan sistem informasi yang dirancang untuk menyajikan data dalam bentuk visual yang ringkas, terintegrasi, dan mudah dipahami oleh pengguna. Dashboard ini umumnya memanfaatkan berbagai komponen visualisasi, seperti grafik, tabel, dan indikator kinerja, untuk menampilkan informasi penting yang mendukung proses pemantauan, analisis, dan pengambilan keputusan. Berbeda dengan laporan statis, dashboard analitik bersifat dinamis dan interaktif, sehingga memungkinkan pengguna untuk memperoleh gambaran kondisi secara cepat dan menyeluruh.

Visualisasi data memiliki peran penting dalam dashboard analitik karena mampu mengubah data mentah menjadi informasi yang lebih bermakna dan intuitif. Melalui visualisasi yang tepat, pengguna dapat dengan mudah mengidentifikasi pola, tren, serta perubahan yang terjadi pada data. Visualisasi data juga berfungsi sebagai sarana penyampaian informasi yang bersifat naratif, di mana kinerja dan kondisi bisnis dapat digambarkan secara komprehensif tanpa harus melalui proses analisis data yang kompleks.

Penelitian yang dilakukan oleh \citep{Rathore2025} menunjukkan bahwa penerapan teknik visualisasi data yang efektif dapat membantu pengambil keputusan dalam menghasilkan keputusan yang lebih informatif, akurat, dan ringkas. Studi tersebut menegaskan bahwa dashboard, berbagai jenis grafik, serta pendekatan visualisasi yang terstruktur berperan penting dalam meningkatkan kualitas business intelligence dan mendukung perencanaan strategis. Selain itu, visualisasi data dinilai mampu meningkatkan efisiensi proses pengambilan keputusan dan mengurangi waktu yang dibutuhkan untuk menganalisis data, sehingga organisasi dapat merespons permasalahan dan dinamika bisnis secara lebih cepat.

Dalam konteks aplikasi data-driven, dashboard analitik berfungsi sebagai jembatan antara data dan pengambilan keputusan. Dashboard memungkinkan penyajian data secara real time atau near real time, sehingga perubahan kondisi dan tren dapat dipantau secara berkelanjutan. Hal ini menjadikan dashboard analitik sebagai komponen penting dalam sistem yang memanfaatkan data berskala besar dan bersifat dinamis, termasuk data yang dihasilkan dari media sosial.

Bagi UMKM, keberadaan dashboard analitik menjadi sangat relevan karena dapat menyederhanakan informasi yang kompleks menjadi tampilan visual yang mudah dipahami. Informasi seperti kecenderungan sentimen konsumen, pola interaksi pengguna, dan ringkasan data pemasaran dapat disajikan secara visual, sehingga pelaku UMKM tidak perlu melakukan analisis data secara manual. Dengan demikian, dashboard analitik dapat mendukung UMKM dalam mengambil keputusan bisnis yang lebih cepat, tepat, dan berbasis data.


\subsection{Arsitektur Client Data Layer}

Pada aplikasi frontend modern yang bersifat data-driven, pengelolaan data yang bersumber dari Application Programming Interface (API) menjadi salah satu aspek penting dalam arsitektur sistem. Aplikasi seperti dashboard analitik umumnya menampilkan berbagai komponen antarmuka, seperti grafik, tabel, dan indikator, yang bergantung pada data yang sama dan diperbarui secara berkala. Jika pengambilan dan pengolahan data dilakukan secara langsung di setiap komponen antarmuka, maka dapat menimbulkan berbagai permasalahan, seperti permintaan API yang berulang, inkonsistensi data antar-komponen, serta peningkatan beban render yang berdampak pada penurunan performa aplikasi.

Client Data Layer merupakan pendekatan arsitektural dalam pengembangan frontend modern yang bertujuan untuk mengelola data yang bersumber dari server secara terpusat di sisi klien. Pendekatan ini muncul sebagai respons terhadap meningkatnya kompleksitas aplikasi data-driven, dimana data bersifat asinkron, dinamis, dan digunakan oleh banyak komponen antarmuka secara bersamaan. Dengan adanya Client Data Layer, proses fetching,chaching, dan synchronize data dapat dilakukan secara lebih terstruktur, sehingga membantu menjaga konsistensi data dan sehingga membantu menjaga konsistensi data dan mendukung pengelolaan data frontend secara lebih terstruktur.

\begin{figure}[H]
  \centering
  \includegraphics[width=0.8\textwidth]{gambar/client-data-layer-paradigm.png}
  \caption{Architecture Client Data Layer}
  \label{fig:client-data-layer-paradigm}
\end{figure}


\begin{packed_enum}
  \item Server State Management

  Server state management mengacu pada pengelolaan data yang bersumber dari sistem eksternal, seperti API atau layanan backend, yang bersifat asinkron dan dapat berubah di luar kendali langsung aplikasi klien. Data ini memiliki karakteristik dinamis karena dipengaruhi oleh kondisi jaringan, waktu respons server, serta pembaruan data di sisi backend. Oleh karena itu, server state memerlukan mekanisme khusus untuk menangani proses pengambilan data, status pemuatan, penanganan kesalahan, serta pembaruan dan sinkronisasi data agar informasi yang digunakan oleh berbagai komponen antarmuka tetap konsisten dan akurat.

  \item Client-side Data Management
  
  Client-side data management berfokus pada pengelolaan data di sisi klien setelah data tersebut diperoleh dari server, termasuk penyimpanan sementara, penggunaan ulang data, serta distribusi data ke berbagai komponen antarmuka. Pendekatan ini bertujuan untuk mengurangi ketergantungan terhadap permintaan data berulang ke server, sehingga dapat mengurangi ketergantungan terhadap permintaan data berulang ke server dan menjaga konsistensi informasi yang ditampilkan. Dengan pengelolaan data yang terstruktur di sisi klien, aplikasi frontend dapat menyajikan data secara lebih responsif dan stabil, khususnya pada aplikasi yang bersifat data-driven seperti dashboard analitik.

  \item Data-driven Frontend Architecture
  
  Data-driven Frontend Architecture merupakan pendekatan pengembangan di mana seluruh antarmuka pengguna didorong oleh data sebagai sumber kebenaran utama. Dalam pola ini, UI dihasilkan sebagai fungsi dari state yang ada—artinya, setiap perubahan pada data akan secara otomatis memicu pembaruan pada tampilan aplikasi. Arsitektur ini sangat berguna untuk aplikasi yang menampilkan informasi dinamis seperti dashboard analitik, papan monitoring, atau aplikasi dengan data real-time. Keunggulan utamanya adalah konsistensi tampilan yang lebih terjamin, alur data yang mudah dilacak, dan pengembangan fitur baru yang lebih sistematis karena UI dan logika data terpisah dengan jelas.

  \item State Management for Asynchronous Data
  
  State Management for Asynchronous Data adalah pendekatan khusus untuk menangani data yang diperoleh melalui state, seperti panggilan API, operasi file, atau permintaan jaringan lainnya. Karena data tersebut tidak tersedia secara instan, sistem harus mampu mengelola berbagai state yang mungkin terjadi: mulai dari state (idle), state fetching atau onloading, state success, hingga state error. Tantangan utamanya adalah memastikan aplikasi tetap responsif dan memberikan umpan balik yang informatif kepada pengguna selama proses pengambilan data. Pendekatan ini juga mencakup strategi seperti pembatalan permintaan yang tidak diperlukan, pengulangan otomatis saat gagal, dan pembaruan data latar belakang untuk menjaga informasi tetap konsisten.

  \item Frontend Data Caching and Synchronization
  
  Data Caching and Synchronization adalah teknik untuk meningkatkan kinerja aplikasi dengan menyimpan salinan data dari server di memori klien. Dengan adanya cache, aplikasi dapat menampilkan informasi secara cepat tanpa perlu melakukan permintaan berulang ke server. Namun, teknik ini juga menimbulkan tantangan, yaitu data yang disimpan dapat menjadi kedaluwarsa jika terjadi perubahan di sisi server. Oleh karena itu, diperlukan mekanisme sinkronisasi yang cerdas, seperti pembaruan di latar belakang (background refresh) dan penandaan cache yang kedaluwarsa (cache invalidation). Dengan pendekatan ini, aplikasi dapat menampilkan data secara lebih konsisten sekaligus menjaga keakuratan informasi yang ditampilkan.

  Beberapa penelitian menunjukkan bahwa penerapan mekanisme caching pada aplikasi frontend dapat meningkatkan performa dan responsivitas sistem. Caching memungkinkan data hasil pemanggilan API disimpan sementara di sisi klien sehingga dapat digunakan kembali tanpa melakukan permintaan ulang ke server. Pendekatan ini terbukti mampu mengurangi duplikasi permintaan API dan mempercepat waktu respons aplikasi. Pemanfaatan pustaka seperti TanStack Query dalam mengelola caching dan prefetching data juga dinilai efektif dalam meningkatkan efisiensi pengelolaan data di sisi frontend serta pengalaman pengguna secara keseluruhan \citep{Rahman2024}. Dalam konteks penelitian ini, peningkatan performa dipahami sebagai perubahan perilaku pengelolaan data frontend yang diamati melalui mekanisme caching dan pengendalian permintaan data, tanpa dilakukan pengukuran performa numerik.

\end{packed_enum}

Penerapan Client Data Layer memberikan sejumlah manfaat dalam pengembangan aplikasi frontend. Salah satu manfaat utama adalah peningkatan konsistensi data, di mana beberapa komponen yang membutuhkan data yang sama dapat memperoleh informasi yang seragam tanpa harus melakukan permintaan data secara terpisah. Selain itu, Client Data Layer memungkinkan pengurangan jumlah permintaan API yang tidak diperlukan melalui mekanisme caching dan pengelolaan siklus data. Pendekatan ini juga mendukung mendukung penyajian data yang lebih stabil dan terkelola serta mempermudah pengelolaan data yang bersifat asinkron dan dinamis.

Dalam konteks dashboard analitik, keberadaan Client Data Layer menjadi semakin penting karena data yang ditampilkan umumnya bersifat besar, sering diperbarui, dan digunakan oleh banyak komponen secara bersamaan. Dengan memanfaatkan Client Data Layer, dashboard dapat menampilkan data secara lebih responsif dan stabil, sekaligus meminimalkan risiko inkonsistensi informasi yang ditampilkan kepada pengguna. Pendekatan ini mendukung terciptanya arsitektur frontend yang lebih terorganisasi, mudah dipelihara, dan skalabel.

\subsection{TanStack Query (React Query)}

TanStack Query merupakan pustaka manajemen data pada sisi frontend yang dirancang untuk mengelola data yang bersumber dari server (server state) secara efisien. Dalam dokumentasi resminya, TanStack Query dijelaskan sebagai “the missing data-fetching layer for web applications” yang berfungsi untuk mempermudah proses pengambilan, penyimpanan sementara (caching), sinkronisasi, serta pembaruan data dari server \citep{TanstackLCC2025}. Pendekatan ini ditujukan untuk menangani kompleksitas data asinkron yang tidak dapat dikelola secara optimal menggunakan mekanisme state management konvensional.

Dalam dokumentasi resminya, TanStack Query mendefinisikan server state sebagai data yang berasal dari sumber eksternal dan memiliki karakteristik asinkron, dapat berubah sewaktu-waktu, serta memerlukan mekanisme khusus untuk menjaga konsistensi data di sisi klien. Oleh karena itu, TanStack Query menyediakan pendekatan deklaratif dalam pengelolaan server state, di mana pengembang dapat mendefinisikan kebutuhan data tanpa harus menangani secara manual proses sinkronisasi dan pembaruan data di setiap komponen antarmuka (TanStack Documentation, 2024).

Salah satu fitur utama TanStack Query adalah mekanisme caching yang memungkinkan data hasil pemanggilan API disimpan sementara di sisi klien. Dengan adanya caching, data yang telah diperoleh dapat digunakan kembali oleh komponen lain tanpa perlu melakukan permintaan ulang ke server, selama data tersebut masih dianggap valid. Pendekatan ini berkontribusi dalam mengurangi jumlah permintaan API yang tidak diperlukan, meningkatkan efisiensi aplikasi, serta mempercepat waktu respons antarmuka pengguna.

Selain caching, TanStack Query juga menyediakan mekanisme sinkronisasi data yang mendukung pembaruan data secara otomatis. Melalui konsep seperti refetching dan invalidasi data, TanStack Query memastikan bahwa data yang ditampilkan tetap mutakhir ketika terjadi perubahan di sisi server. Mekanisme ini sangat relevan pada aplikasi data-driven, seperti dashboard analitik, yang menampilkan data secara dinamis dan digunakan oleh banyak komponen secara bersamaan.

Dalam konteks arsitektur frontend, TanStack Query dapat diposisikan sebagai implementasi konkret dari Client Data Layer. Pustaka ini berperan sebagai lapisan perantara antara backend API dan komponen antarmuka pengguna, sehingga komponen UI tidak berinteraksi langsung dengan API. Dengan demikian, TanStack Query membantu memisahkan logika pengelolaan data dari logika tampilan, meningkatkan keterbacaan kode, serta mempermudah pemeliharaan aplikasi dalam jangka panjang.

Pada penelitian ini, TanStack Query digunakan sebagai solusi untuk menerapkan arsitektur Client Data Layer pada pengembangan dashboard analitik. Fokus penggunaan TanStack Query diarahkan pada pengelolaan server state, caching data, serta sinkronisasi data antar-komponen, tanpa membahas aspek internal pustaka atau detail implementasi secara mendalam. Dengan pendekatan ini, TanStack Query berperan sebagai fondasi pengelolaan data pada sisi frontend yang mendukung penyajian informasi sentimen secara konsisten, responsif, dan efisien.

\subsection{REST API}

Representational State Transfer Application Programming Interface (REST API) merupakan gaya arsitektur layanan web yang digunakan untuk memungkinkan komunikasi antara klien dan server melalui protokol HTTP secara terstandarisasi. REST API bersifat stateless, di mana setiap permintaan dari klien harus membawa seluruh informasi yang dibutuhkan untuk diproses oleh server, sehingga tidak bergantung pada status permintaan sebelumnya. Data yang dipertukarkan umumnya disajikan dalam format JSON karena bersifat ringan dan mudah diproses oleh aplikasi frontend, menjadikan REST API banyak digunakan pada aplikasi web modern yang bersifat data-driven.

Dalam konteks aplikasi frontend analitik, REST API berperan sebagai sumber utama data (server state) yang dikonsumsi oleh antarmuka pengguna. Data yang diperoleh melalui REST API bersifat asinkron dan dapat berubah sewaktu-waktu, sehingga memerlukan mekanisme pengelolaan data yang mampu menangani proses pengambilan, pembaruan, dan sinkronisasi data secara efisien. Penelitian terkini menunjukkan bahwa REST API memiliki keunggulan dalam penyajian data yang bersifat datar dan mudah di-cache, sehingga mampu meningkatkan efisiensi distribusi data serta memperbesar rasio cache hit pada lapisan jaringan. Pendekatan ini dinilai lebih optimal untuk kebutuhan aplikasi yang menampilkan data terstruktur secara berulang, seperti dashboard dan sistem pelaporan, dibandingkan pendekatan API lain yang lebih kompleks \citep{Islam2025}.

Lebih lanjut, penelitian tersebut menegaskan bahwa performa aplikasi web tidak ditentukan oleh satu teknologi tertentu, melainkan oleh keselarasan antara desain akses data, mekanisme caching, serta pengelolaan state pada sisi klien. REST API yang dirancang dengan kontrak data yang jelas dan cache-aware terbukti mendukung peningkatan performa dan skalabilitas aplikasi ketika dipadukan dengan lapisan pengelolaan data di frontend. Oleh karena itu, dalam pengembangan aplikasi frontend modern, REST API umumnya tidak diakses secara langsung oleh setiap komponen antarmuka, melainkan melalui lapisan pengelolaan data seperti Client Data Layer agar data dapat dikelola secara terpusat, konsisten, dan efisien.

Dalam penelitian ini, REST API diposisikan sebagai penyedia data hasil analisis sentimen yang diproses di sisi backend. Data tersebut selanjutnya dikelola pada sisi frontend melalui arsitektur Client Data Layer sebelum ditampilkan dalam bentuk visualisasi pada dashboard analitik. Dengan pemisahan peran ini, REST API berfungsi sebagai sumber data, sementara pengelolaan performa, caching, dan sinkronisasi data dilakukan sepenuhnya di sisi frontend untuk mendukung penyajian informasi yang responsif dan konsisten.

\section{Penelitian Terkait}

Berikut adalah tabel perbandingan penelitian terkait yang relevan dengan pengembangan penerapan arsitektur Client Data Layer menggunakan TanStack Query pada dashboard sentiment analysis

\begin{table}[H]
\centering
\caption{Tabel Perbandingan Penelitian Terkait}
\label{tab:perbandingan_penelitian}
\scriptsize
\setlength{\tabcolsep}{4pt} % mengurangi padding agar muat
\renewcommand{\arraystretch}{1.2}
\begin{tabular}{|c|p{2.8cm}|p{2.5cm}|p{4cm}|p{4.5cm}|}
\hline
\textbf{No} &
\textbf{Peneliti} &
\textbf{Teknologi} &
\textbf{Judul} &
\textbf{Fitur} \\
\hline

1 &
Fajarini, Sri Dwi; Kurniawati, Juliana; Yuliani, Fitria (2025) &
NLP, Machine Learning (SVM, Random Forest, VADER) &
\textit{Social Media Sentiment Analysis as a New Tool for Predicting Market Trends and Consumer Behaviour} &
Analisis sentimen media sosial untuk mengidentifikasi pola opini publik dan memprediksi perilaku konsumen serta tren pasar. \\
\hline

2 &
Shrutika Rathore; Rahul Nawkhare; Navin Sharma; Nitin Chaudhary; Saurabh Chakole; Bhaskar Vishwakrama (2025) &
Dashboard analitik, visualisasi data, business intelligence &
\textit{Effective Data Visualization Techniques for Business Decision-Makers} &
Penyajian data bisnis melalui dashboard analitik untuk meningkatkan efisiensi pengambilan keputusan dan perencanaan strategis. \\
\hline

3 &
Micheal, Author Dave (2024) &
React, React Query (TanStack Query), lazy loading, caching &
\textit{React Query and Lazy Loading: Performance Optimization Best Practices} &
Optimasi performa aplikasi frontend melalui pengelolaan data asinkron, caching, dan lazy loading untuk mengurangi permintaan API berulang. \\
\hline

\end{tabular}
\end{table}

\section{Analisa Gap Penelitian}

\subsection{Gap Penelitian 1}

Penelitian ini berfokus pada pemanfaatan analisis sentimen media sosial menggunakan pendekatan Natural Language Processing (NLP) dan machine learning untuk memprediksi perilaku konsumen dan tren pasar. Namun demikian, penelitian ini belum membahas bagaimana hasil analisis sentimen tersebut dikelola dan disajikan pada sisi frontend, khususnya dalam bentuk dashboard analitik. Aspek arsitektur pengelolaan data frontend, seperti mekanisme pengambilan data dari API, caching, serta sinkronisasi data antar-komponen, belum menjadi fokus dalam penelitian ini.

\subsection{Gap Penelitian 2}

Penelitian ini menekankan pada peran dashboard dan teknik visualisasi data dalam meningkatkan efektivitas pengambilan keputusan bisnis. Fokus utama penelitian berada pada desain visualisasi, jenis grafik, serta manfaat dashboard analitik bagi pengambil keputusan. Namun, penelitian ini belum mengkaji bagaimana data yang ditampilkan pada dashboard dikelola di sisi frontend, khususnya terkait arsitektur pengambilan data, pengelolaan server state, serta mekanisme caching dan konsistensi data pada aplikasi frontend modern.

\subsection{Gap Penelitian 3}

Penelitian ini membahas penggunaan React Query dalam pengelolaan data asinkron pada aplikasi React, serta pemanfaatan lazy loading dan caching untuk meningkatkan performa aplikasi frontend. Meskipun penelitian ini menunjukkan bahwa React Query efektif dalam mengurangi pemanggilan API berulang dan meningkatkan waktu respons aplikasi, konteks penerapannya masih bersifat umum. Penelitian ini belum membahas penerapan React Query sebagai bagian dari arsitektur Client Data Layer pada aplikasi dashboard analisis sentimen, serta belum mengaitkan pengelolaan server state dengan kebutuhan penyajian data analitik pada konteks UMKM.